V tejto diplomovej práci sme sa venovali návrhu riešenia na predspracovanie mračien bodov vysokej hustoty, pričom sme sa snažili znížiť ich pamäťovú náročnosť a zároveň zachovať dôležité body, ktoré opisovali detaily. Zo spracovaného mračna bodov sme následne vytvorili tzv. mesh, ktorá opisovala body spojitým povrchom, ktorý bol na záver otextúrovaný. 
\newline\indent V prvej časti práce sme sa venovali zhrnutiu existujúcich metód pre spracovanie mračien bodov, ako aj pre rekonštrukciu povrchu. Venovali sme všeobecným prístupom, ale aj fungovaniu konkrétnych algoritmov, čo nám umožnilo hlbší náhľad pri ich využití.
\newline\indent V ďalšej časti práce sme sa zamerali na samotný návrh riešenia, pričom pridávanie jednotlivých krokov, bolo usmernené priebežnými výsledkami. Začali sme filtráciou mračna bodov od odľahlých bodov využitím \acrshort{sor} metódy, čo do istej úrovne zabezpečilo odstránenie falošných bodov, ako aj menej hustých oblastí, ktoré neboli podstatné pre konečný výsledok. Pokračovali sme využitím progresívneho morfologického filtra, ktorý nám oddelil body zeme a objektov. Z oddelených bodov objektov sme následne vytvorili tzv. supervoxely, ktoré spájali podobné body do väčších celkov a boli predpokladom pre využitie \acrshort{cpc} segmentačnej metódy. Táto metóda nám umožnila prístup ku jednotlivým objektom, čo sme využili pri ďalších krokoch.
\newline\indent Dôležitým krokom bolo podvzorkovanie, ktoré výrazne znížilo celkový počet bodov. Pre body objektov sme využili adaptívne podvzorkovanie založené na metóde diferencií normál, ktorou sme klasifikovali dôležitosť jednotlivých bodov a následne sme využili príslušnú úroveň podvzorkovania. Využitím Poissonovej rekonštrukčnej metódy, sme vytvorili povrch, ktorý pri menšej pamäťovej náročnosti opisoval pôvodné mračno bodov. Na záver sme vytvorený povrch otextúrovali pomocou fotiek z kamery, čo obohatilo výsledok o realizmus.
\newline\indent Celé riešenie sme na overili pomocou vizuálnej kontroly a porovnaní, pričom sme odhalili niekoľko nedostatkov, ktoré ale vo väčšine prípadov nemali závažný dopad na výsledok. Z dosiahnutých výsledkov môžeme povedať, že uvedený postup funguje najlepšie v prípade štrukturovaných prostredí z nízkou úrovňou šumu, zatiaľ čo v opačnom prípade sa znižuje efektivita podvzorkovania a dôveryhodnosť rekonštruovaného povrchu.
\newline\indent Napriek dosiahnutým výsledkom má táto práca stále mnoho miesta pre vylepšenie. Mnoho využitých metód v tejto práci je založených na nastavení parametrov, ktoré môžu  fungovať optimálne pre jeden súbor dát, ale nemusia pre druhý. Bolo by preto vhodné doplniť dynamické nastavovanie niektorých parametrov na základe vlastností konkrétneho súboru dát, a taktiež by bolo možné doplniť medzikrok, ktorý by zvýšil úroveň odolnosti na šum.
