Diplomovej práca sa zaoberá problematikou mračien bodov a ich efektívnym uchovaním prostredníctvom rekonštrukcie povrchu. Úvod práce je venovaný preskúmaniu existujúcich metód a algoritmov na spracovanie mračien bodov, ako aj na rekonštrukciu povrchu. Cieľom práce je zníženie pamäťových nárokov na uchovanie 3D informácií o nasnímanom priestore, pričom sa snažím zachovať detaily. Naša metóda začína očistením mračna bodov od odľahlých bodov, čím zachová iba relevantné informácie. Pokračuje separáciou bodov zeme a objektov, zhlukovaním bodov do supervoxelov a ich následnou segmentáciou, ktorá umožní individuálny prístup ku jednotlivým objektom. Ďalším krokom je adaptívne podvzorkovanie bodov objektov, po ktorom je vykonaná samotná rekonštrukcia povrch pomocou Poissonovej rekonštrukčnej metódy. Práca bola overená na niekoľkých súboroch dát získaných v reálnych podmienkach, pričom dosahovala vysoké zníženie pamäťových nárokov. 