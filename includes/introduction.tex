\noindent V posledných rokoch prudký nárast vývoja technológii a výpočtovej techniky viedol ku zvýšeniu využívania 3D dát vo forme mračien bodov, ktoré sú často získavané pomocou techník ako \acrshort{lidar}, fotogrametrie alebo stereo kamery. Mračná bodov dokážu veľmi presne zachytávať geometrickú reprezentáciu rôznorodých prostredí, čo ich robí dobrou voľbou v oblastiach ako sú robotika, mapovanie prostredí, autonómne vozidlá a medzi inými aj vo výrobe.
\newline\indent Mračná bodov obsahujú milióny bodov a v neupravenom stave často predstavujú výzvu z hľadiska efektívneho ukladania, spracovania a prístupu ku požadovanej časti mračna. Jedným spôsobom na odstránenie tohto problému je zníženie celkového počtu bodov pomocou podvzorkovania, nakoľko vo väčšine prípadov bývajú mračná bodov príliš husté a tú istú informáciu zachytávajú niekoľko krát. Týmto spôsobom znížime pamäťové a výpočtové nároky, ale pri bežných metódach sa stretneme buď s nízkou úrovňou podvzorkovania alebo stratou detailov. Je preto potrebné navrhnúť adaptívny prístup podvzorkovania, ktorý bude brať do úvahy dôležitosť bodov.
\newline\indent Samotné mračná bodov predstavujú iba jednotlivé body objektov a po aplikácií podvzorkovania môže dôjsť ku problematickej reprezentácií pôvodného povrchu. Je preto potrebné z mračna bodov vytvoriť povrch v podobe tzv. mesh-u, ktorý bude reprezentovať body pomocou sady polygónov, a zároveň s tým výtvory štruktúru medzi bodmi.
\newline\indent V tejto práci sa preto budeme zaoberať preskúmaním existujúcich metód a algoritmov pre spracovanie mračien bodov ako aj rekonštrukciu povrchu. Následne navrhneme naše riešenie, ktoré spracuje mračno bodov, vytvorí štrukturovanú reprezentáciu jeho povrchu a zároveň zníži celkové pamäťové nároky.