\noindent V posledných rokoch prudký nárast vývoja výpočtovej techniky a cenové sprístupnenie technológií, viedlo ku zvýšeniu využívania 3D dát vo forme mračien bodov, ktoré sú často získavané pomocou techník, ako sú \acrshort{lidar}, fotogrametria alebo stereo kamera. Mračná bodov dokážu veľmi presne zachytávať geometrickú reprezentáciu rôznorodých prostredí, čo ich robí dobrou voľbou v oblastiach, ako sú robotika, mapovanie prostredí a medzi inými aj vo výrobe.
\newline\indent Mračná bodov často obsahujú milióny bodov a v neupravenom stave predstavujú výzvu z hľadiska efektívneho ukladania, spracovania a prístupu ku požadovanej časti priestoru. Jedným spôsobom pre odstránenie tohto problému, je zníženie celkového počtu bodov pomocou podvzorkovania, nakoľko vo väčšine prípadov bývajú mračná bodov príliš husté a tú istú informáciu zachytávajú niekoľko krát. Týmto spôsobom znížime pamäťové a výpočtové nároky, ale pri bežných metódach sa stretneme buď s nízkou úrovňou podvzorkovania alebo stratou detailov. Je preto potrebné navrhnúť adaptívny prístup podvzorkovania, ktorý bude brať do úvahy dôležitosť jednotlivých bodov.
\newline\indent Samotné mračná bodov predstavujú iba jednotlivé body povrchu a po aplikovaní podvzorkovania, môže dôjsť ku problematickej reprezentácií pôvodných objektov. Je preto potrebné, vytvoriť z mračna bodov povrch v podobe tzv. mesh-u, ktorý bude reprezentovať prepojenia bodov pomocou sady polygónov, čím odstráni vzniknuté diery. Týmto spôsobom sa taktiež vytvorí efektívnejšia štruktúra, ktorá pri menšej pamäťovej náročnosti bude zachytávať ten istý povrch.
\newline\indent V tejto práci sa preto budeme zaoberať preskúmaním existujúcich metód a algoritmov pre spracovanie mračien bodov, ako aj rekonštrukciu povrchu. Následne navrhneme naše riešenie, ktoré spracuje mračno bodov, vytvorí reprezentáciu jeho povrchu a zároveň zníži celkové pamäťové nároky.