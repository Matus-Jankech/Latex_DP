The diploma thesis deals with the issue of point clouds and their efficient storage through surface reconstruction. The introduction of the thesis is dedicated to exploring existing methods and algorithms for point cloud processing, as well as surface reconstruction. The aim of the thesis is to reduce the memory requirements for storing 3D information about the scanned space, while preserving details. Our method begins by cleaning the point cloud from outlier points, retaining only relevant information. It continues with the separation of ground and object points, clustering points into supervoxels, and their subsequent segmentation, which allows individual access to each object. The next step is adaptive subsampling of object points, followed by the surface reconstruction itself using the Poisson reconstruction method. The work was verified on several datasets obtained under real conditions, while achieved a significant reduction in memory requirements.